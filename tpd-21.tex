\documentclass[10pt,journal,compsoc]{IEEEtran}

\usepackage{verbatim}
\usepackage{url}
\usepackage{graphicx}

\usepackage{xspace}
\newcommand{\FC}       {Freechains\xspace}
\newcommand{\reps}     {\emph{reps}\xspace}
\newcommand{\onerep}   {\emph{1~rep}\xspace}
\newcommand{\nreps}[1] {\emph{#1~reps\xspace}}

\newcommand{\Xon} {$1{\rightarrow}N$\xspace}
\newcommand{\Xno} {$1{\leftarrow}N$\xspace}
\newcommand{\Xnn} {$N{\leftrightarrow}N$\xspace}
\newcommand{\Xoo} {$1{\leftrightarrow}1$\xspace}
\newcommand{\Xo}  {$1{\hookleftarrow}$\xspace}

\begin{document}
\title{
    Peer-to-Peer Consensus via Authoring Reputation
}
\author{
    %Francisco Sant'Anna~\IEEEmembership{Department of Computer Science, Rio de Janeiro State University}
}

\IEEEtitleabstractindextext{%
\begin{abstract}
Content publishing in public Internet forums suffers from excess and abuse,
such as SPAM and fake news.
Centralized platforms employ filtering algorithms and anti-abuse policies, but
impose full trust from users.
We propose a publish-subscribe peer-to-peer protocol to model content
dissemination without centralized control.
The protocol prevents Sybil attacks with a reputation system that moderates
content and, at the same time, delivers network consensus.
We trace a parallel with Bitcoin:
    posts create reputation (vs proof-of-work),
    likes and dislikes transfer reputation (vs transactions),
    and aggregate reputation determines consensus (vs longest chain).
The reputation system and resulting consensus depends exclusively on human work
to create and rate content.
We prototype a simple permissionless distributed version control system that
relies on reputation consensus to resolve conflicts automatically.
\end{abstract}

\begin{IEEEkeywords}
Bitcoin, CRDT, distributed consensus, peer-to-peer, publish-subscribe, reputation system, version control system
\end{IEEEkeywords}}

\maketitle

% TOTAL: 12 pages

\section{Introduction}
\label{sec.introduction}

\IEEEPARstart{C}{ontent} publishing in public Internet forums and social media
platforms is increasingly more centralized in a few
companies~\cite{internet.fixing,p2p.osn}.
%
On the one hand, these companies offer free storage, friendly user interfaces,
and robust access.
On the other hand, they concentrate more power than required to operate, since
they collect and control our data, ``algorithmize'' our consumption, and yet
obstruct portability with proprietary standards.
%
Peer-to-peer alternatives~\cite{p2p.survey} eliminate intermediaries and push
to end users the responsibility to manage data and connectivity.
However, due to decentralization of authority and infrastructure, new
challenges arise to enforce overall state consistency while dealing with
malicious users.

In an ideal Internet forum, all messages or posts
(i)   reach even temporarily disconnected users;
(ii)  are delivered in a consistent order;
(iii) are respectful and on topic.
In a centralized system, items (i) and (ii) are trivially achieved assuming
availability and delivery order in the service, while for item (iii), users
have to trust the service to moderate content.
In a decentralized setting, however, none of these demands are easily
accomplished.
A common approach in gossiping protocols is to proactively replicate and
disseminate conversations in peers until they reach all
users~\cite{p2p.survey}.
However, this approach does not guarantee consensus since posts can be received
in conflicting orders~\cite{p2p.intention}.
%As an example, antagonistic messages such as \emph{"X is final"} vs
%\emph{"Y is final"} might be sent concurrently, preventing the network to
%determine as a group its intention as \emph{X} or \emph{Y}.

Bitcoin~\cite{p2p.bitcoin} proposes a permissionless consensus protocol founded
on scarce virtual assets, the \emph{bitcoin tokens}.
%
The only way to create new bitcoins is to work towards consensus in the network
by proposing a total order among transactions in the system.
%
This way, Bitcoin prevents double spending~\cite{p2p.bitcoin}, which is
analogous to conflicting message delivery in public discussions.
    %deciding between \emph{X} and \emph{Y} as a group is the same as
    %transferring bitcoins to \emph{X} and \emph{Y} with insufficient funds for
    %both.
%
However, Bitcoin just supports transfers between users, with no subjective
judgment that could affect the actual transactions.
In contrast, our goal is to use social interactions between humans to evaluate
content and mitigate abuse at the same time.

In this work, we propose a permissionless consensus algorithm based on
authoring reputation.
Inspired by Bitcoin, authors accumulate tokens named \reps, which serve as
currency to rate posts in the forums.
Users can rate posts with likes and dislikes, which transfer \reps between
them.
Work is manifested as new posts which, if accepted by others, reward authors
with \reps.
This way, like Bitcoin, token generation is expensive, while verification is
cheap and made by multiple users.
However, unlike Bitcoin, both creation and verification are subjective, based
on human creativity and judgement, which match our target domain of content
publishing.
Posts and likes are linked as blocks in a Merkle~DAG that persists the whole
conversation and is disseminated in the network with gossiping.
To reach consensus, the DAG is ordered by branches with more reputed authors,
which contributed with more work to the forum.
The resulting list is then verified for conflicting operations, such as likes
with insufficient \reps, which is equivalent to double spending in Bitcoin.
In this case, the branch that causes the conflict is removed from the DAG.
%
We integrated the proposed consensus algorithm into Freechains, a peer-to-peer
publish-subscribe content dissemination protocol~\cite{fcs.sbseg20}.
We also prototyped a permissionless distributed version control system (DVCS)
that relies on consensus to apply automatic merges.

\begin{figure*}
\centering
\includegraphics[width=\textwidth]{arrangements.png}
\caption{The three types of chains and arrangements in \FC.}
\label{fig.table}
\end{figure*}

Our main contribution is to make public forums with complete decentralization
viable in practice.
The proposed reputation and consensus mechanism depends exclusively on human
work, contrasting with most systems that rely on extrinsic resources, such as
CPU power.
The general idea of the algorithm can be applied to any system that uses DAGs
to structure its messages.
As a derived contribution, the consensus implies a total order among messages,
which backs the use of CRDTs~\cite{p2p.merkle-crdts} in collaborative authoring
platforms, such as a DVCS.

As a main limitation, Merkle~DAGs are ever growing data structures that also
carry considerable metadata overhead.
In addition, the required per-block validation may become a bottleneck for
real-time applications, such as chats and video calls.
%
Finally, we do not claim that the proposed reputation system enforces ``good''
human behavior in any way.
Instead, it provides a transparent and quantitative mechanism that helps users
understand the progress of forums and act accordingly.

In Section~\ref{sec.freechains}, we introduce the basic functionalities of \FC
to create, rate, and disseminate posts.
In Section~\ref{sec.consensus}, we describe the general reputation and
consensus mechanism, and show how it integrates with public forums in \FC.
In Section~\ref{sec.crdts}, we discuss the some correspondences with CRDTs and
prototype a simple DVCS.
In Section~\ref{sec.related}, we compare our system with other
publish-subscribe protocols, federated applications, and fully
peer-to-peer systems.
In Section~\ref{sec.conclusion}, we conclude this work.

\section{Freechains}
\label{sec.freechains}

\FC is an unstructured peer-to-peer topic-based publish-subscribe system, in
which each topic or \emph{chain} is a replicated
\emph{Merkle~DAG}~\cite{fcs.sbseg20}.
This way, as an author posts to a chain, other users subscribed to the same
chain eventually receive the message.
\FC supports multiple arrangements of public and private communication, which
are detailed in Figure~\ref{fig.table}.
In this section, we operate a private group to describe the basic behavior of
chains.
At the end of the section, we also exemplify a public identity chain.
In Section~\ref{sec.consensus}, we detail the behavior of public forums, which
involve untrusted communication between users and require the proposed
reputation and consensus mechanism.

All \FC operations go through a \emph{daemon} (analogous to Bitcoin full nodes)
which validates posts, links them in the Merkle~DAGs, persists the chains in
the disk, and communicates with other peers to disseminate the graphs.
The command that follows starts a daemon to serve further operations:

{\footnotesize
\begin{verbatim}
 > freechains-daemon start '/var/freechains/'
\end{verbatim}
}

The actual chain operations use a separate client to communicate with the
daemon.
The next sequence of commands (i) creates a shared key, (ii) joins a private
group chain (prefix $\$$), and (iii) posts a message into the chain:

{\footnotesize
\begin{verbatim}
 > freechains key shared 'strong-password' # (i)
 A6135D..   <- returned shared key
 > freechains '$family' join 'A6135D..'       # (ii)
 42209B..   <- hash of chain
 > freechains '$family' post 'Good morning!'  # (iii)
 1_EF5DE3.. <- hash of post
\end{verbatim}
}

\begin{figure*}
\centering
\includegraphics[width=\textwidth]{family.png}
\caption{
    Three DAG configurations.
    (A) Single head pointing to genesis block.
    (B) Fork with heads pointing to genesis block.
    (C) Like pointing to previous heads and also to its target.
}
\label{fig.family}
\end{figure*}

A private chain requires that all participants use the same shared key to join
the group.
A \emph{join} only initializes the DAG locally in the file system, and a
\emph{post} also only modifies the local structure.
No communication occurs at this point.
Figure~\ref{fig.family}.A depicts the state of the chain after the first post.
The genesis block with height $0$ and hash \texttt{42209B..}
depends only on the arguments given to \emph{join}.
The next block with height $1$ and hash \texttt{EF5DE3..} contains the posted
message.
As expected from a Merkle~DAG, the hash of a block depends on its payload and
hash of previous block.

\FC adheres to the \emph{local-first} software principle~\cite{p2p.local},
allowing networked applications to work locally while offline.
Except for synchronization, all other operations in the system affect only the
local replica.
In particular, joining a chain with the same arguments in another peer results
in the same genesis state, even if the peers have never met before.
Hence, before synchronizing, others peers have to initialize the example chain
with the same steps:

{\footnotesize
\begin{verbatim}
 > freechains-daemon start '/var/freechains/'
 > freechains key shared 'strong-password'
 A6135D..
 > freechains '$family' join 'A6135D..'
 42209B..
\end{verbatim}
}

Synchronization is explicit, in pairs, and unidirectional.
The command \emph{recv} asks the daemon in \emph{localhost} to connect to
daemon in \emph{remote-ip} and receive all missing blocks from there:

{\footnotesize
\begin{verbatim}
 > freechains '$family' recv '<remote-ip>'
 1/1  <- one block received from <remote-ip>
\end{verbatim}
}

Now, the new peer is in the same state as the original peer in
Figure~\ref{fig.family}.A.
The complementary command \emph{send} would synchronize the DAGs in the other
direction.
Note that \FC does not construct a network topology nor synchronizes peers
automatically.
There are no preconfigured peers, no root servers, no peer discovery.
All connections happen through the \emph{send} and \emph{recv} commands which
have to specify the peers explicitly.
In this sense, the protocol only gives basic support for communication in pairs
of peers and further automation requires external tools.

In order to query the state of the replica, the next sequence of commands
checks the hash(es) of the block(s) at the head of the local DAG (the latest
blocks), and then reads the payload of the single head found:

{\footnotesize
\begin{verbatim}
 > freechains '$family' heads
 1_EF5DE3..
 > freechains '$family' payload '1_EF5DE3..'
 Good morning!
\end{verbatim}
}

However, since the network is inherently concurrent and users are encouraged to
work locally, typical graphs are not lists, but DAGs with multiple heads.
As an example, suppose the new peer posted a message before the \emph{recv}
above, when the local DAG was still in its genesis state.
In this case, as illustrated in Figure~\ref{fig.family}.B, the resulting graph
after the synchronization would now contain two blocks with height $1$.
%
Note that forks in the DAG create ambiguity in the order of messages, which is
a fundamental obstacle to reach consensus.
In private chains, we can apply simple methods, such as relying on the
timestamps of blocks.
However, in public forums, a malicious user could modify his local time to
manipulate the order of messages.

To conclude the basic chain operations, users can rate posts with \emph{likes}
and \emph{dislikes}, which can be consulted later:

{\footnotesize
\begin{verbatim}
 > freechains '$family' like '1_EF5DE3..'
 2_BF3319..
 > freechains '$family' reps '1_EF5DE3..'
 1  <-- post received 1 like
\end{verbatim}
}

As illustrated in Figure~\ref{fig.family}.C, a like is a regular block with an
extra link to its target.
In private groups, likes are unlimited and behave much like typical centralized
systems.
In public forums, however, likes are restricted, have to be signed by users,
and are at the core of our proposed consensus algorithm.

For the sake of completeness, \FC also supports public identity chains (prefix
$@$) with owners attached to public/private keys:

{\footnotesize
\begin{verbatim}
 > freechains key pubpvt 'other-password'
 EB172E.. 96700A..   <- public and private keys
 > freechains '@EB172E..' join
 F4EE21..
 > freechains '@EB172E..' post 'This is Oprah' \
    --sign='96700A..'
 1_547A2D..
\end{verbatim}
}

In the example, a public figure creates a key pair and joins an identity chain
attached to her public key.
Every post in the chain needs to be signed with her private key to be accepted
in the network.

\FC is around $1500$ LoC in Kotlin and is publicly available%
\footnote{\url{http://www.freechains.org}}.
The binary for the JVM is less than $6Mb$ in size and works in Android and most
desktop systems.

\section{Reputation and Consensus Mechanism}
\label{sec.consensus}

In the absence of moderation, permissionless peer-to-peer public forums are
impractical.
At the root of the problem lies Sybil attacks, which use large numbers of fake
identities to abuse the system.
For instance, it should take a few seconds to generate thousands of
public/private key identities and SPAM million of messages into the system.
%Hence, without moderation, there are no limits on the number and size of posts
%and no reasonable policy to distinguish quality.
For this reason, we propose a reputation system that works together with a
consensus algorithm to mitigate Sybil attacks and make peer-to-peer public
forums practical.

Section~\ref{sec.consensus.design} describes the overall reputation and
consensus mechanism, which can be applied to any public forum system that uses
DAGs to structure its messages.
Section~\ref{sec.consensus.chains} describes the concrete rules we implement
for public forums in \FC.

\subsection{Overall Design}
\label{sec.consensus.design}

In the proposed reputation system, users can spend tokens named \reps to post
and rate content in the forums:
a \emph{post} initially penalizes authors until it consolidates and counts
positively;
a \emph{like} is a positive feedback that helps subscribers distinguish good
content amid excess;
a \emph{dislike} is a negative feedback that revokes content when crossing a
threshold.
Figure~\ref{fig.general} summarizes the reputation operations and their goals.
%
However, without restrictions, posts and likes alone are not satisfactory in
the presence of Sybils.
Therefore, \reps must be subject to some sort of scarcity that demands
non-trivial work immune to automation.

\begin{figure}
\centering
\includegraphics[width=0.49\textwidth]{general.png}
\caption{General reputation operations in public forums.}
\label{fig.general}
\end{figure}

Bitcoin employs CPU proof-of-work to mitigate Sybil attacks.
However, CPU or alternative extrinsic resources are not evenly distributed
among humans, specially considering that most communications use
battery-powered devices.
%
In the context of content publishing, we understand that authoring ability is
already an intrinsic human resource that we can take advantage.
Creating new content is hard and takes time, but is comparatively easy to
verify and rate.
Therefore, in order to impose scarcity, we determine that only content
authoring generates \reps, while likes and dislikes just transfer \reps between
users.
%but limited by periods of time to prevent excess (e.g., once in a day per user).
%Additionally,
%
Still, scarce posts and likes are not yet sufficient because they demand
consensus in the network.
As an example, it is possible that an author with a single unit of \reps
receives a dislike at the same time she tries to post a new message in the
network.
If accounted before, the dislike blocks the new post, otherwise the post is
valid.
Therefore, we need the same message ordering across all peers to account the
reputation of users and validate the operations consistently.

\begin{figure}
\centering
\includegraphics[width=0.49\textwidth]{reps2.png}
\caption{
    (A) A public forum DAG with a common prefix and two branches.
    (B) Total order between blocks of the DAG.
}
\label{fig.reps}
\end{figure}

Our solution is to order posts favoring forks with participants that constitute
the majority of reputation in the network.
This way, in order to manipulate accountability, malicious users first need to
cooperate to gain reputation, which is non-trivial and contradicts their
intent.

Figure~\ref{fig.reps}.A illustrates the reputation criterion.
%, still abstractly, since we did not discuss the actual rules for content creation and rating.
A public forum DAG has a common prefix with signed posts from users $a$, $b$,
and $c$.
Let's assume that within the prefix, users $a$ and $b$ have contributed with
better content and have more reputation combined than $c$ has alone.
%
After the prefix, the forum forks in two branches:
in \emph{branch~1}, only user $c$ remains active and we see that new users $x$
and $y$, with no previous reputation, generate a lot of new content;
in \emph{branch~2}, only users $a$ and $b$ participate but with less activity.
Nonetheless, \emph{branch~2} takes priority because, before the forking point,
$a$ and $b$ have more reputation than $c$, $x$, and $y$ combined.
%
User $c$ represents here a malicious user trying to cultivate fake identities
$x$ and $y$ in separate of the network to accumulate \reps.
However, the whole malicious \emph{branch~1} is vulnerable because users in
\emph{branch~2} with more previous reputation take the priority and can
overthrow user $c$.

Figure~\ref{fig.reps}.B indicates the consensus order between blocks in the
forum.
All operations in \emph{branch~2} are considered before any operation in
\emph{branch~1}.
The consensus ordered list is only a view of the primary forum DAG structure
for accountability purposes.
At any point in the consensus timeline, if an operation fails, all remaining
blocks in the offending branch are removed from the primary DAG.
As an example, suppose the last post by $a$ (in gray) is a dislike to user $c$,
which decreases its reputation.
Then, it's possible that the last post by $c$ (in red) is rejected together
with all posts by $y$ and $x$ in sequence.
%
Note that in a Merkle~DAG, it is not possible to remove only the block with the
failing operation, instead, we need to remove the whole remaining branch as if
it never existed.
%
Note also that users in the branch with more reputation may react to attacks
even after the fact.
For instance, users $a$ and $b$ can pretend that they did not yet see
\emph{branch-1} and post extra dislikes to user $c$ from \emph{branch-2} so
that a further merge with \emph{branch-1} removes all of its blocks.

Some other considerations about forks and merges:
%
Peers that received branches with less reputation first (\emph{branch~1}) will
need to reorder all blocks starting at the forking point.
This might even involve removing content in the end user software.
This behavior is similar to blockchain reorganization in Bitcoin when a peer
detects a new longest chain and disconsiders old blocks.
%
Likewise, peers that saw branches with more reputation first (\emph{branch~2})
just need to put the other branch in sequence and do not need to recompute
anything.
This should be the normal behavior and is expected to happen in the majority of
the network.
%
Unlike Bitcoin, forks are not only allowed but encouraged due to the
local-first software principle.
However, the longer a peer remains disconnected, the more conflicting
operations it may perform, and the higher are the chances of rejection when
rejoining.

\begin{figure}
\centering
\includegraphics[width=0.49\textwidth]{merge.png}
\caption{
    Merging rules:
    (a) The branch with more reputation in the common prefix is ordered first.
    (b) A branch with 50\% or more reputation then its prefix is ordered first.
    (c) The merge fails if rules (a) and (b) conflict.
}
\label{fig.merge}
\end{figure}

\begin{figure*}
\centering
\includegraphics[width=\textwidth]{rules.png}
\caption{Specific reputation rules for public forum chains in \FC.}
\label{fig.rules}
\end{figure*}

As a counterpoint, suppose users $a$ and $b$ in Figure~\ref{fig.reps}.A have
actually abandoned the chain for months and thus \emph{branch-1} is legit.
In this case, $a$ and $b$ might be the ones trying to take over the chain.
A third possibility is that both branches are legit but became disconnected for
a long period.
It is simply impossible to determine.
Nonetheless, it is unacceptable to permit that a very old branch affects a long
active chain.
For this reason, the consensus algorithm includes an extra constraint when
merging:
If a branch creates enough \reps to reach $50\%$ of its prefix, then the
algorithm preserves this branch as first in future merges.
In the example, suppose that the common prefix accumulates \nreps{50}
considering users $a$, $b$, and $c$.
If \emph{branch-1} creates at least $25$ new \reps, then the merge with
\emph{branch-2} will fail and the chains will never synchronize again.
This situation is analogous to a hard fork in Bitcoin.
%
Figure~\ref{fig.merge} summarizes the merging algorithm:
    rule (a) favors branches with more reputation;
    rule (b) preserves branches with $50\%+$ \reps as first;
    rule (c) enforces that rules (a) and (b) do not conflict.

A fundamental drawback of Merkle~DAGs is that all replicas in the system need
to store the complete graph in order to synchronize and verify new blocks.
Tree pruning techniques allow to remove parts of the graph to save
space~\cite{p2p.prune}.
The rule (b) in the consensus algorithm allows to prune the chain DAG when
crossing the $50\%+$ threshold, at least for lightweight clients in
resource-constrained devices.
%A branch that reaches the threshold of $50\%+$ \reps is freezed in the ordered
%list and becomes safe to remove along with all of its past back to the genesis
%block.
However, these devices can no longer verify older forks and need to delegate
trust to more powerful peers.
%
A more pragmatic approach, but which requires cooperation among users, is to
revoke past posts (rule~\texttt{3.b}) which will delete payloads and save some
space.
This approach is more feasible in private groups and public identitis chains,
tough.

\subsection{Public Forum Chains}
\label{sec.consensus.chains}

We integrated the proposed reputation system in the public forums of \FC to
support content moderation and enforce consensus in the chains.
Figure~\ref{fig.rules} details the concrete rules which are discussed as
follows.
Authors have to sign their posts in order to be accounted by the reputation
system and operate in the chains.
The example that follows creates an identity whose public key is assigned as
the pioneer in a public chain (prefix $\#$):

{\footnotesize
\begin{verbatim}
 > freechains key pubpvt 'pioneer-password'
 4B56AD.. DA3B5F..
 > freechains '#forum' join '4B56AD..'
 10AE3E..
 > freechains '#forum' post --sign='DA3B5F..' \
    'The purpose of this chain is...'
 1_CC2184..
\end{verbatim}
}

The \emph{join} command in rule~\texttt{1.a} bootstraps a public chain,
assigning \nreps{30} equally distributed to the pioneers referred in the public
keys.
The pioneers shape the initial culture of the chain with their first posts and
likes, while they gradually transfers \reps to other authors, which may also
transfer to other authors, expanding the community.
%
In this regard, the \emph{post} command in sequence above, which is signed by
the single pioneer, indicates the purpose of the chain for future users.

The most basic concern in public forums is to resist Sybils spamming the
chains.
Fully peer-to-peer systems cannot rely on identity logins or CAPTCHAs due
to the lack of a central authority.
Other alternatives include (i) building social trust graphs, in which users
already in the community vouch for new users, or (ii) imposing economic costs
for new posts, such as proof of work.

We propose a mix between trust graphs and economic costs.
%
Rule~\texttt{4.a} imposes that authors require at least \onerep to post,
effectively blocking Sybil actions.
To vouch for new users, rule~\texttt{3.a} allows that an existing user likes a
newbie's post to unblock it, but at the cost of \onerep.
This cost prevents that malicious members unblock new users indiscriminately,
which would be a breach for Sybils.
For the same reason, rule~\texttt{2} imposes a temporary cost of \onerep for
each new post.
%
Note that the pioneer rule~\texttt{1.a} solves the chicken-and-egg problem
imposed by rule~\texttt{4.a}.

In the next sequence of commands, a new user joins the same public chain and
posts a message, which is welcomed with a like signed by the pioneer:

{\footnotesize
\begin{verbatim}
 > freechains key pubpvt 'new-author-password'
 503AB5.. 41DDF1..
 > freechains '#forum' join '4B56AD..'
 10AE3E..     <-- same pioneer as before
 > freechains '#forum' post 'Im a newbie...' \
    --sign='41DDF1..'
 2_C3A40F..   <-- blocked post
 > freechains '#forum' like '2_C3A40F..' \
    --sign='DA3B5F..'
 3_59F3E1..
\end{verbatim}
}

Note that chains with the same name but different pioneers are incompatible
because the hash of genesis blocks also depend on the pioneers' public keys.

Figure~\ref{fig.forum} illustrates the chain DAG up to the like operation.
The pioneer starts with \nreps{30} (rule~\texttt{1.a}) and posts the initial
message.
%
New posts penalize authors with \nreps{-1} during at most 12 hours
(rule~\texttt{2}), which depends on the activity succeeding (and including) the
new post.
The more activity from reputed authors, the less time the discount persists.
In the example, since the post is from the pioneer controlling all \reps in the
chain, the penalty falls immediately and she remains with \nreps{30}.
This mechanism limits the excess of posts in chains dynamically.
For instance, in slow technical mailing lists, it is more expensive to post
messages in sequence.
However, in chats with active users, the penalty can decrease to zero.

\begin{figure}
\centering
\includegraphics[width=0.49\textwidth]{forum.png}
\caption{
    The \texttt{like} approves the newbie message into the \texttt{\#forum} DAG.
}
\label{fig.forum}
\end{figure}

Back to Figure~\ref{fig.forum}, a new user with \nreps{0} tries to post a
message (hash~\texttt{C3A40F..}) and is blocked (rule~\texttt{4.a}), as the red
background highlights.
Then, the pioneer likes the blocked message, decreasing herself to \nreps{29}
and increasing new user to \onerep (rule~\texttt{3.a}).
Note that the newbie post is not penalized because it is followed by the
pioneer, which still controls all \reps in the chain (rule~\texttt{2}).

With no additional rules to generate \reps, the initial \nreps{30} would
constitute the whole ``chain economy'' forever.
For this reason, rule~\texttt{1.b} awards authors of new posts with \onerep,
but only after 24 hours.
This rule stimulates content creation and grows the economy of chains.
The 24-hour period gives sufficient time for other users to judge the post
before awarding the author.
It also regulates the growth speed of the chain.
In Figure~\ref{fig.forum}, after 1 day, the pioneer now accumulates \nreps{30}
and the new user \nreps{2}, growing the economy in \nreps{2} as result of the
two consolidated posts.
Note that rule~\texttt{1.b} awards at most one post at a time.
Hence, new posts during the 24-hour period will not award extra \reps to the
author.
Note also that rule~\texttt{4.b} limits authors to at most \nreps{30}, which
provides incentives to spend likes and thus decentralize the network.

Likes and dislikes (rules \texttt{3.a} and \texttt{3.b}) serve three purposes
in the chains:
    (i) welcoming new users,
    (ii) measuring the quality of posts, and
    (iii) censoring abuse (SPAM, fake news, illegal content, etc).
%
Access to chains is permissionless in the sense that the actual identities
behind posts are irrelevant for acceptance.
Instead, it is the quality of content that is verified and accounted in the
system.
%
The reputation of a given post is the difference between its likes and
dislikes, which can be used in end-user software for filtering and highlighting
purposes.
%
The quality of posts is subjective and is up to users to judge then with likes,
dislikes, or simply abstaining.
%
On the one hand, since \reps are finite, users need to ponder to avoid
indiscriminate expenditure.
On the other hand, since \reps are limited to at most \nreps{30} per author
(rule~\texttt{4.b}), users also have incentives to rate content.
Hence, the upper and lower limits work together towards the quality of the
chains.
%
Note that a dislike shrinks the chain economy since it removes \reps from both
the origin and target.
As detailed next, the actual contents of a post may become hidden if it has at
least 3 dislikes, and more dislikes than likes (rule~\texttt{3}).
However, considering that \reps are scarce, dislikes should be more directed to
combat abuse, but not much to eliminate divergences of opinion.

\begin{figure}
\centering
\includegraphics[width=0.49\textwidth]{state.png}
\caption{
    State machine of posts:
    \emph{BLOCKED} posts are not linked in the DAG.
    The payload of \emph{HIDDEN} posts are not retransmitted.
    \emph{ACCEPTED} posts are linked and retransmitted.
}
\label{fig.state}
\end{figure}

A post has three possible states: \emph{BLOCKED}, \emph{ACCEPTED}, or
\emph{HIDDEN}.
Figure~\ref{fig.state} specifies the transitions between states.
%
If the author has reputation, a new post is immediately \emph{ACCEPTED} in the
chain.
Otherwise, it is \emph{BLOCKED} and requires a like from another user.
Blocked posts are not considered part of the chain DAG in the sense that new
posts do not link back to it.
%
Peers are not required to hold blocked posts and neither retransmit them to
other peers.
However, if blocked posts do not reach other users, they will never have the
chance to be welcomed with a like.
A reasonable policy is to hold blocked posts in a temporary bag and retransmit
them for some visibility in the network.
Rule~\texttt{4.c} limits the size of posts to at most \emph{128Kb} to prevent
DDoS attacks using gigantic blocked posts.
%
Once accepted, a post becomes part of the chain and can never be removed
again, since Merkle~DAGs are immutable by design.
%Note that blocked posts that become accepted are always succeeded by a
%\emph{like} (Figure~\ref{fig.forum}).
%
If the number of dislikes exceeds the threshold (rule~\texttt{3}), the block
becomes \emph{HIDDEN} and its payload is not retransmitted to other peers.
The hash of the block depends only on the hash of the payload, so it is safe to
remove the actual payload as long as you can prove its hidden state.
Later, if the post receives new likes and changes its state again, it means
that the payload is still known somewhere and peers can request it when
synchronizing again.

\section{Correspondence with CRDTs}
\label{sec.crdts}

Conflict-free replicated data types (CRDTs)~\cite{p2p.crdts} serve as a robust
foundation to model shared data with concurrent updates in networked
local-first applications~\cite{p2p.local}.

At the transport layer, a Merkle~DAG chain is itself a trivial state-based CRDT
(CvRDT) because, on synchronization, the missing parts of its self-verifiable
DAG are exchanged to converge to the same state~\cite{p2p.merkle-crdts}.
%
At the application layer, however, DAGs are not CRDTs because branches are
delivered to peers in different orders, and thus can lead to different states
when processed.
%
An interesting approach is to require blocks in the DAGs to represent
commutative operations, which lead to operation-based CRDTs (CmRDTs), now at
the application layer~\cite{p2p.merkle-crdts}.
%
CmRDTs only requires to store the update operations in order to reconstruct any
version of the data.
In contrast, CvRDTs would require to store each complete version of the data
for the same purpose.
%
Our proposed consensus algorithm goes one step further and transforms a chain
DAG into a totally-ordered set, which leads to a CRDT that does not require
commutative operations.

These mechanisms lead to three-layered CRDT scheme to build distributed
applications:
    a CvRDT at the transport layer,
    a CmRDT relying on commutative operations at the application layer, and
    a CRDT with arbitrary operations after consensus is applied.
Next, we illustrate this scheme through an example of a simple permissionless
DVCS implemented on top of public chains.

\subsection{A DVCS with Automatic Merge}

We built a simple DVCS with the operations as follows:
%
\begin{itemize}
    \setlength{\itemindent}{-8pt}
    \item initialize a repository (\texttt{join})
    \item commit local changes to repository (\texttt{post})
    \item checkout repository to local (\texttt{traverse/payload})
    \item synchronize with remote peer (\texttt{send/recv})
    \item rate commits with likes \& dislikes (\texttt{like/dislike})
\end{itemize}
%
The associated \FC commands appear in parenthesis.
%
The system relies on public forum chains, which means that repositories are
permissionless and adhere to the reputation and consensus mechanism.
The main innovations in this system are that
    (i)  users can rate commits to reject them, and
    (ii) checkout operations merge commits automatically based on consensus.

As a concrete example, we model a Wiki article as a chain behaving as a VCS to
hold its full edition history.
An article can refer to other articles using hyperlinks to other chains.

Most VCS operations, except \emph{commit} and \emph{checkout}, map directly to
single \FC operations.
For instance, to create a repository, we simply join a public chain with the
name of the file we want to track.
In the example that follows, we create a repository with multiple pioneers, and
then edit and commit the file multiple times:

{\footnotesize
\begin{verbatim}
 > freechains #p2p.md join A2885F.. 2B9C32..
 > echo "Peer-to-peer networking is..." > p2p.md
 > freechains-vcs #p2p.md commit --sign=699299..
 1_4F3EE1..
 > echo "The [USENET](#usenet.md), ..." >> p2p.md
 > freechains-vcs #p2p.md commit --sign=699299..
 2_B58D22..
\end{verbatim}
}

The \emph{commit} operation expands as follows:

{\footnotesize
\begin{verbatim}
 > freechains-vcs #p2p.md checkout p2p.remote
 > diff p2p.remote p2p.md > p2p.patch
 > freechains #p2p.md post p2p.patch --sign=699299..
\end{verbatim}
}

It first makes a temporary checkout to file \texttt{p2p.remote}.
Then, it diffs the remote and the local versions with our changes to file
\texttt{p2p.patch}.
Finally, the commit operation posts the signed patch file back into the chain.
Of course, the chain name and signature are parameters in the actual
implementation.

Much time later, the other pioneer synchronizes with us, checks out the file,
and then edits and commits it back:

{\footnotesize
\begin{verbatim}
 > freechains #p2p.md recv '<our-ip>'
 2/2    <-- two commits above
 > freechains-vcs #p2p.md checkout p2p.md
 > echo "Peer-to-Peer does not scale!" >> p2p.md
 > freechains-vcs #p2p.md commit --sign=320B59..
 23_AE3A1B..
\end{verbatim}
}

The checkout operation needs to recreate the file by applying all patches since
the genesis block:

{\footnotesize
\begin{verbatim}
 > rm p2p.md && touch p2p.md
 > for blk in `freechains #p2p.md traverse` do
     freechains #p2p.md payload $blk > p2p.patch
     patch p2p.md p2p.patch
     if [ $? != 0 ]; then
       echo $blk  <-- hash of failing patch
       break
     fi
   done
\end{verbatim}
}

The \texttt{traverse} operation of \FC returns all hashes since the genesis
block respecting the consensus order.
The loop reads each of the payloads representing the patches and apply them in
order to recreate the newest version of the file in the repository.
If the patch fails, the command exhibits the hash of the block in the screen
and terminates.

Since the last commit above is clearly wrong, other users in the network will
dislike it until the block becomes \emph{HIDDEN} in the chain (as described in
Figure~\ref{fig.state}):

{\footnotesize
\begin{verbatim}
 > freechains #p2p.md dislike 23_AE3A1B.. --sign=USR1
 > freechains #p2p.md dislike 23_AE3A1B.. --sign=USR2
 > freechains #p2p.md dislike 23_AE3A1B.. --sign=USR3
 > freechains-vcs #p2p.md checkout p2p.md
 > cat p2p.md
 Peer-to-peer networking is...
 The [USENET](#usenet.md), ...
 ...
\end{verbatim}
}

This way, the checkout operation above will apply an empty patch associated
with the hidden block, removing the offending from the file.
This mechanism illustrates how the reputation system enables collaborative
permissionless editing.

Next, we create a conflicting situation in which two authors edit and commit
the same line of the file concurrently:

{\footnotesize
\begin{verbatim}
 # PEER A (more reputation):
 > sed -i 's/peer/Peer/g' p2p.md    <-- fix typo
 > freechains-vcs #p2p.md commit --sign=699299..
 27_A..

 # PEER B (less reputation):
 > sed -i 's/networking/computing/g' p2p.md
 > freechains-vcs #p2p.md commit --sign=320B59..
 27_B..

 # SYNCHRONIZE (exchange conflicting forks):
 > freechains #p2p.md recv '<our-ip>'
 1 / 1
 > freechains #p2p.md send '<our-ip>'
 1 / 1
 > freechains-vcs #p2p.md checkout p2p.md
 1 hunk FAILED -- saving rejects to file p2p.md.rej
 27_B..
 > cat p2p.md
 Peer-to-Peer networking is...      <-- typo fixed
 The [USENET](#usenet.md), ...
 ...
\end{verbatim}
}

After they commit the conflicting changes, the peers synchronize in both
directions and reach the state of Figure~\ref{fig.conflict}.
When we checkout the file, the patches are applied respecting the consensus
order.
As a result, although the conflict exists, we see that the first branch is
applied, but not the second, leaving the file in the longest possible
consistent state.
%
Because of the \texttt{break} in the checkout operation above, note that once a
conflict is found, no further patches apply in any of the remaining branches.
%
We chose to adopt a \emph{first write wins} resolution to favor work in the
branches with more reputation.
Nonetheless, the failing patch branch is not totally ignored, since the
checkout saves the conflict file and indicates the block causing it.
%
We believe the optimistic choice that does not reject both patches is the most
advantageous, because it keeps the file in an usable state while still showing
that there is a conflict to resolve.
For instance, the authors can later decide to dislike one of the two commits to
settle the file (possibly inverting their reputation) and remove the warning.

\begin{figure}
\centering
\includegraphics[width=0.35\textwidth]{conflict.png}
\caption{
    The branches in the DAG are ordered by reputation.
    Only the first patch is applied successfully (first write wins).
}
\label{fig.conflict}
\end{figure}

\subsection{Discussion}

In summary, the proposed reputation and consensus mechanism empowers a simple
DVCS with cooperative authoring and automatic conflict resolution.
It only requires \emph{diff \& patch} and the basic API of \FC with no access
to the internal structure of DAGs.

We apply the three-layered CRDT scheme as follows:
In the first layer, the whole file commit history of small patches is
transported as a DAG between the peers with the basic commands of \FC.
%Eventually, all peers reach same state with the full commit history.
At this layer, the DAG is just raw data with no attached semantics.
However, in order to recreate the file, peers need to interpret the DAG as
small editions, which requires a commutative operation in the second layer.
The \emph{patch} tool is mostly commutative, except when branches modify the
same lines.
Hence, in these situations we resort to the total order in the third layer,
which we apply sequentially to patches until a conflict occurs.
The final state of the file is guaranteed to be consistent, i.e., the result of
a sequence of correct patch applications.

Applications that rely only on commutative operations do not require to
traverse the DAG in the same order.
This is the case of most social apps with threaded conversations, in which
branches typically do not interfere with each other.
For instance, it is not problematic to rely on block timestamps to display
messages in chats, forums, and social media posts.
This is also the general case for notifications in these applications, such as
status updates and social engagements.

Towards distributed collaborative applications, we can employ richer CRDT data
models other than raw text.
As an example, \emph{Automerge}~\cite{p2p.automerge} is a CRDT to manipulate
JSON objects, which provides non-trivial datasets with robust merging policies.

\section{Related Work}
\label{sec.related}

% Content Dissemination
% CRDTs
% Consensus

Many other systems have been proposed for distributed content
dissemination~\cite{p2p.survey,p2p.ecosystem}.
Here we consider publish-subscribe protocols, federated applications, and fully
peer-to-peer systems.

\subsection{Publish-Subscribe Protocols}

Decentralized topic-based publish-subscribe protocols, such as
    \emph{XMPP}~\cite{pubsub.xmpp},
    \emph{ActivityPub}~\cite{pubsub.activitypub}, and
    \emph{gossipsub}~\cite{pubsub.gossipsub},
decouples publishers from subscribers in the network.
%
A key limitation of \emph{pubsubs} is that the brokers or servers that mediate
communication still have a special role in the network.
They serve the communication queues and may, as examples, authenticate and
validate posts.
%
Nonetheless, some pubsubs do not rely on server roles and, instead, use gossip
dissemination in unstructured peer-to-peer networks~\cite{pubsub.tera,pubsub.rappel,pubsub.stan,pubsub.vitis,pubsub.gossipsub,pubsub.rappel}.
Most of these protocols focus on techniques to achieve scalability and
performance, such as throughput, simultaneous connections, load balancing, and
real-time relaying.
However, these techniques alone are not sufficient to operate permissionless
networks that have to deal with malicious nodes, particularly
Sybils~\cite{pubsub.gossipsub2}.

As generic protocols, pubsubs are typically unaware of the applications built
on top of them.
In contrast, the pubsub of \FC integrates with the semantics of chains and
verifies blocks during connections.
For instance, to flood the network with posts, malicious nodes need to spend
reputation, which takes hours to recharge (rule~\texttt{2} in
Figure~\ref{fig.rules}).
Blocked posts (Figure~\ref{fig.state}) have limited reachability and are not a
concern either.
Another advantage of a tight integration is that the Merkle~DAG structure
simplifies the synchronization between nodes, provides persistence, and
prevents duplication of messages.
Full persistence facilitates dealing with long churn periods, and
de-duplication tolerates CmRDTs with operations that are not idempotent.

\subsection{Federated Applications}

E-mail is the most popular federated application, allowing users from one
domain to seamlessly exchange messages with users of other domains.
More recently, \emph{Diaspora}, \emph{Matrix}, and \emph{Mastodon} filled the
domains of social media, chat, and microblogging~\cite{p2p.ecosystem}.

As a drawback, identities in federations are not portable across domains, which
may become a problem when servers shutdown or users become unsatisfied.
In any of these cases, users have to grab their content, move to another
server, and announce a new identity to followers.

Moderation is also a major concern in federated
applications~\cite{p2p.ecosystem}.
As an example, messages crossing domain boundaries may be subject to different
policies that might affect delivery.
With no coordinated consensus, it is difficult to make pervasive public forums
practical in federated applications.
%
Matrix supports a moderation system%
\footnote{Matrix moderation: \url{https://matrix.org/docs/guides/moderation}},
but which is permissioned and applied only after messages have already been
disseminated in the network.

As a counterpoint, federated protocols seem to be more appropriate for
stream-based real-time applications such as large chats rooms.
The number of hops and header overhead can be much smaller in client-server
architectures compared to peer-to-peer systems, which typically include message
signing, hash linking, and extra verification rules.

\subsection{Peer-to-Peer Systems}

Bitcoin~\cite{p2p.bitcoin} is probably the most successful permissionless
network but serves specifically for electronic cash.
IPFS~\cite{p2p.ipfs} and Dat~\cite{p2p.dat} are data-centric systems for
hosting large files and applications, respectively.
Scuttlebutt~\cite{p2p.scuttlebutt} and Aether~\cite{p2p.ecosystem} are closer
to \FC and cover human-centric \Xon and \Xnn public communication, respectively.

Bitcoin adopts CPU \emph{proof-of-work} to achieve consensus, which does not
solve the centralization issue given the high costs of equipment and energy.
\emph{Proof-of-stake} is a prominent alternative~\cite{p2p.proofs} that
acknowledges that centralization is inevitable (i.e., the richer gets richer)
and uses a function of time and wealth to elect nodes to mint new blocks.
As an advantage, these proof mechanisms are generic and apply to multiple
domains, since they depend on an extrinsic scarce resource.
%
In contrast, we chose an intrinsic resource, which is authoring content for the
chains themselves.
We believe that human work grows linearly with effort and is not portable
across chains with different topics.
This hypothesis supports the intended decentralization.
%
Another distinction is that generic public ledgers typically require permanent
connectivity to avoid forks, which opposes our local-first principle.
This is because a token transaction only has value as part of the longest
chain.
This is not the case for a local message communication, which has value in
itself.

IPFS~\cite{p2p.ipfs} is centered around immutable content-addressed data, while
Dat~\cite{p2p.dat} around mutable pubkey-addressed data.
IPFS is more suitable to share large and stable content such as movies and
archives, while Dat is more suitable for dynamic content such as web apps.
%
Both IPFS and Dat use DHTs as their underlying architectures, which are optimal
to serve large and popular content, but not for search and discovery.
In both cases, users need to know in advance what they are looking for, such as
the exact link to a movie or a particular identity in the network.
%
On the one hand, DHTs are probably not the best architecture to model
decentralized human communication with continuous feed updates.
On the other hand, holding large files in Merkle~DAGs replicated across the
network is also impractical.
An alternative is to use DHT links in Merkle payloads to benefit from both
architectures.

Scuttlebutt~\cite{p2p.scuttlebutt} is designed around public identities that
follow each other to form a graph of connections.
This graph is replicated in the network topology as well as in data storage.
For instance, if identity $A$ follows identity $B$, it means that the computer
of $A$ connects to $B$'s in a few hops and also that it stores all of his posts
locally.
This architecture is very similar to \Xon public identity chains of \FC in
Figure~\ref{fig.table}.
%
For group \Xnn communication, Scuttlebutt uses the concept of channels, which
are in fact nothing more than hash tags (e.g. \emph{\#sports}).
Users can tag posts, which appear not only in their feeds but also in virtual
feeds representing these channels.
However, users only see channel posts from users they already follow.
In practice, channels simply merge friends posts and filter them by tags.
In theory, to read all posts of a channel, a user would need to follow all
users in the network (which also implies storing their feeds).
A limitation of this model is that new users struggle to integrate in channel
communities because their posts have no visibility at all.
As a counterpoint, channels are safe places that do not suffer from abuse.
%
In \FC new users require a single like for visibility in the community, which
relies on the reputation system to prevent abuse.
Also, a chain stores its own posts only, instead of unrelated posts from its
subscribers.

Aether~\cite{p2p.ecosystem} provides peer-to-peer public communities aligned
with \Xnn public forums of \FC.
A fundamental difference is that is designed for ephemeral, mutable posts with
no intention to enforce global consensus across peers.
Aether employs a very pragmatic approach to mitigate abuse in forums, using
established techniques, such as proof-of-work to combat SPAM, and an innovative
voting system to moderate forums, but which affects local instances only.
In contrast, \FC relies on its reputation and consensus mechanisms for
moderation.

\section{Conclusion}
\label{sec.conclusion}

We propose a new permissionless consensus algorithm for content dissemination
in peer-to-peer networks.
The main insight of the algorithm is to use the human authoring ability as a
scarce resource to determine consensus.
This contrasts with extrinsic resources, such as CPU power, which are
dispendious and not evenly distributed among people.% willing to communicate.

Consensus is backed by a reputation system in which users can rate posts with
likes and dislikes, which transfer reputation between them.
The only way to forge reputation is authoring new content, which is subject to
judgement by other users.
This way, reputation generation is expensive, while verification is cheap.

The reputation and consensus mechanism is integrated into \FC, a peer-to-peer
protocol that offers permissionless public forums that are immune to Sybil
attacks.
The protocol replicates Merkle~DAGs that represent causal relationships between
messages in the network.
The consensus algorithm transforms DAGs into totally-ordered sets, which
eliminates conflicts analogous to double spending in Bitcoin.
The mechanism can be applied to any system that uses DAGs to structure its
messages.

We also conceptualize a three-layered CRDT scheme to build collaborative
applications:
    (i)   the Merkle~DAG as a CvRDT at the transport layer,
    (ii)  commutative updates as a CmRDT at the application layer,
    (iii) arbitrary operations as a CRDT after consensus is applied.
%
On top of this scheme, we built a simple distributed version control system
that resolves commit conflicts automatically.

\begin{comment}

where else can we use this?
    - reddit

- Users can fork or recreate the chain.
  Unlike bitcoin, the value is not on the size, but the cohesion is users.
- Safe for niche topics and minority groups                                     
- Fulfill the expectations of part of the community                             
- Part from the principle that the pioneers want it to decentralize, otherwise would create a public identity chain fig2

- INSIGHT
 Uso descentralizado
 Emissão descentralizada e restrita
 Criação difícil / Verificação fácil

- users that work twice, bots

- summary
    - parallel w/ bitcoin
        - ponte entre reputacao -> sybil -> consenso
    - no problem w/ forks
    We reach a similar solution to Bitcoin but adapted to our domain
    - we need consensus to make the reputation system work
    - we need reputation system to reach consensus
    - same virutous cycle as bitcoin
    - The more CPU work is done, the stronger becomes the proposal, the more peers
        follow it, the more tokens are mined.
        There is a strong association between work, profit and consensus that enables
        Bitcoin as a peer-to-peer cash system.
    - reputation system to rate messages
        - positively (to distinguish from excess), or
        - negatively (to block SPAM, fake news, illegal content)
    - some sort of scarcity (work)
        - you like you loose
        - you work you get
        - otherwise sybil, "likes" abuse
        - incentives for continuous, good quality posts
    - Like \emph{bitcoins}, \reps are scarce, hard to generate, and easy to verify.
      Unlike \emph{bitcoins}, \reps .

\end{comment}

\begin{comment}
- content                                                                       
- CPU work to create blocks, easy objective verification, 50+1 attack, collapse
- Human work to create content, easy subjective verification, 50+1 attack,
  community fork (actually encouraged)
- unique identity based on CPU
- here based on post quality
- not CDN: Content delivery network                                             

(b) double spend of coins/reps is solved by total ordering all
Users can like \& dislike posts, which transfer reputation between them.
Reputation is created from news

Just like Bitcoin reaches consensus with the longest chain
uses mining to

(excess, SPAM, fake, abuse, illegal)                                            
\end{comment}

\bibliographystyle{IEEEtran}
\bibliography{tpd-21}

\begin{comment}
\begin{IEEEbiography}[{\includegraphics[width=1in,height=1.25in,clip,keepaspectratio]{chico}}]{Francisco Sant'Anna}
received his PhD degree in Computer Science from PUC-Rio, Brazil in
2013.
In 2016, he joined the Faculty of Computer Science at the Rio de Janeiro State
University, Brazil.
His research interests include Programming Languages and Concurrent \&
Distributed Systems.
\end{IEEEbiography}
\end{comment}

\end{document}
