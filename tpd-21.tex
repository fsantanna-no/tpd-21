\documentclass[10pt,journal,compsoc]{IEEEtran}

\usepackage{verbatim}
\usepackage{url}

\begin{comment}
- content                                                                       
- CPU work to create blocks, easy objective verification, 50+1 attack, collapse
- Human work to create content, easy subjective verification, 50+1 attack,
  community fork (actually encouraged)
- unique identity based on CPU
- here based on post quality
- not CDN: Content delivery network                                             

(b) double spend of coins/reps is solved by total ordering all
Users can like \& dislike posts, which transfer reputation between them.
Reputation is created from news

Just like Bitcoin reaches consensus with the longest chain
uses mining to

(excess, SPAM, fake, abuse, illegal)                                            
\end{comment}

\begin{document}
\title{
    Peer-to-Peer Consensus based on Content Reputation
}
\author{
    Francisco Sant'Anna~\IEEEmembership{Department of Computer Science, Rio de Janeiro State University}
}

\IEEEtitleabstractindextext{%
\begin{abstract}
Content publishing in social media and public Internet forums are subject to
excess and abuse such as low quality posts and fake news.
Centralized platforms employ filtering algorithms and anti-abuse policies, but
demand full trust from users.
We propose a publish-subscribe peer-to-peer protocol to model public content
dissemination without centralized moderation.
The central idea of the protocol is a reputation system that serves to moderate
content and, at the same time, reach network consensus.
We trace a parallel with Bitcoin:
    consolidated posts create reputation (vs proof-of-work);
    likes and dislikes transfer reputation (vs transactions);
    forks are ordered by reputation (vs longest chain).
The reputation system prevents content excess and abuse, imposes consensus on
a peer-to-peer setting, and depends solely on human work to create and rate
content.
\end{abstract}

\begin{IEEEkeywords}
peer-to-peer, consensus, reputation system, publish-subscribe
\end{IEEEkeywords}}

\maketitle

\section{Introduction}
\label{sec.introduction}

\IEEEPARstart{C}{ontent} publishing , such as in public Internet forums and
social media platforms, is increasingly more centralized in the hands of a few
companies~\cite{internet.fixing}.
%
On the one hand, these companies offer friendly user interfaces, free storage,
and permanent connectivity.
On the other hand, they concentrate more power than required to operate, since
they control our data, collect private information, tune our consumption based
on algorithms, and yet obstruct portability with proprietary protocols.
%
Peer-to-peer alternatives~\cite{p2p.survey} eliminate intermediates and push to
end users the responsibility to manage data and connectivity.
Because of the decentralization of authority and network infrastructure, some
new challenges arise, such as dealing with malicious users, ensuring continuous
content discovery, and enforcing overall state consistency.

In a scenario of interest, suppose users want to discuss a political event by
posting public comments in the network.
In an ideal system,
(i)   all posts would eventually reach all users, even those temporarily
      disconnected;
(ii)  posts would be delivered to users in a consistent order;
(iii) the interactions among users would be respectful and on topic.
In a centralized system, items (i) and (ii) are trivially achieved assuming
availability and delivery order in the service, while for item (iii) users must
trust the service to moderate content, e.g., removing SPAM and fake news.
In a decentralized setting, however, none of these demands are easily
accomplished.
A common approach in gossiping protocols~\cite{p2p.survey} is to replicate the
whole conversation in all peers and disseminate proactively until all users
receive it.
However, this approach does not guarantee consensus since messages can be
received in conflicting orders in different peers.
As an example, antagonistic messages such as \emph{"X is decided"} and
\emph{"Y is decided"} might be sent concurrently and the network as a whole
cannot decide between \emph{X} or \emph{Y}.

Bitcoin~\cite{p2p.bitcoin} is the first successful permissionless consensus
protocol, allowing anyone to participate in the network while still being
resistant to Sybil attacks.
%
As an inspiration to our work, Bitcoin is centered around the concept of
\emph{bitcoin tokens}, which are scarce internal assets that can be transferred
between users.
The only way to create new tokens is to work towards consensus in the network
by proposing a total order among all transactions in the system.
The more work is done, the stronger becomes the proposal, the more peers follow
it, the more tokens are acquired.
There is a strong association between work, profit and consensus that enables
Bitcoin as a peer-to-peer cash system.
%
As its main contribution, Bitcoins prevents the double-spending problem as a
consequence of the total order achieved with consensus~\cite{p2p.bitcoin}.
This problem is the same we want to solve for public discussions:
deciding between \emph{X} or \emph{Y} as a group (when only one option is
applicable) is the same as buying \emph{X} or \emph{Y} (when funds are
insufficient for both).
%
However, Bitcoin's basic operation is to blindly transfer tokens between users,
with no subjective judgment that could affect transactions.
Considering our scope of social interactions between humans, we aspire for a
decentralized and qualitative evaluation of content in the network.

In this work, we propose a consensus algorithm based on the reputation of
content created by humans.
Inspired by Bitcoin, users accumulate tokens named \emph{reps}, which serves as
currency to rate users and posts in the network.
Users can rate posts with likes (and dislikes), which transfer \emph{reps}
among them.
In the algorithm, work is manifested with new posts, which reward authors with
\emph{reps}, but which are still judged by other users.
This way, like Bitcoin, token generation is (typically) expensive, while
verification is cheap and made by multiple users.
However, unlike Bitcoin, both creation and verification are subjective, based
on human creativity and judgement, which match our target domain of content
publishing.

Consensus is still required because like operations spend \emph{reps} that need
to be verified consistently by all peers in the network.
Posts and likes are linked as nodes in a Merkle~DAG that persists the whole
conversation and is disseminated in the network.
To verify operations in concurrent branches, the graph must be sequenced to
generate a total order between nodes.
Our sequencing proposal is straightforward and compares the sum of \emph{reps}
accumulated in the past by the authors in pairs of concurrent branches.
The branch with more reputation---the one with more work---prefixes the
sequence as a whole.
If the suffixed branch has a conflicting operation (e.g., a like from a user
without reputation), then this operation and all remaining nodes are removed
from the DAG.
The proposed consensus algorithm is integrated in Freechains%
\footnote{\url{http://www.freechains.org}}~\cite{fcs.sbseg20},
a peer-to-peer content dissemination protocol.

In Section~\ref{sec.freechains}, we introduce the basic functionality of
the peer-to-peer protocol.
In Section~\ref{sec.consensus}, we describe the consensus algorithm.
- goals, incentives, possible attacks and mitigations
In Section~\ref{sec.related},
In Section~\ref{sec.conclusion},

\section{Freechains}
\label{sec.freechains}

\section{Consensus Algorithm}
\label{sec.consensus}

\section{Related Work}
\label{sec.related}

\section{Conclusion}
\label{sec.conclusion}



With this design, we retarget  , the balance between work, profit and consensus also applies 

attack 50+1 but not as
cite local-first software

 (crypto proof \& authoring),
while verification is cheap


requires work but evaluation


, but
are evaluated by other user
The system also imposes incentives to rate posts and 

Unlike bitcoin

Since likes and dislikes spend reputation, \emph{creps} are scarce resources
Incentives to post and rate content.

There is a strong association between work, profit and consensus that enables
Bitcoin as a peer-to-peer cash system.

distinguish SPAM from legitimate
Another problem CPU



content.



 (e.g., posts on social media, public conversations
based on the reputation of users
in the network.


biggest difference:
work is subjective as is the evaluation by other users


 with consensus since
whoever proposes the ordering will choose one of the operations arbitrarily


 which is similar to the
example above (\emph{buy X} vs \emph{buy Y})

spent

with work



this gives consensus with total order, solves double spend, which is equivalent
to solving X/Y is decided above

we borrow token, scarcity


 that requires work to 
Only one purpose


- incentive
- security

- Last-Write-Wins

- just all tokens are the same, no subjective judgment, for example on why token is being transferred


 and xx timestamps.


 of the service and , and trust from users to deal

In a decentralized setting, 
    - spam
    - abuse
    - on topic
    - disconnections
    - order

Bitcoin

responses would




n user posts 

- discovery
- also consensus, ensure that participants receive all data in a consistent order
- bitcoin, discovery even disconnected, consensus, but not quality


Regardless of the Internet growth over the years,

\subsection{Subsection Heading Here}
Subsection text here.

\subsubsection{Subsubsection Heading Here}
Subsubsection text here.

\section{Conclusion}
The conclusion goes here.

\bibliographystyle{IEEEtran}
\bibliography{tpd-21}

\begin{IEEEbiography}{Michael Shell}
Biography text here.
\end{IEEEbiography}

\begin{IEEEbiographynophoto}{John Doe}
Biography text here.
\end{IEEEbiographynophoto}

\begin{IEEEbiographynophoto}{Jane Doe}
Biography text here.
\end{IEEEbiographynophoto}

\end{document}
